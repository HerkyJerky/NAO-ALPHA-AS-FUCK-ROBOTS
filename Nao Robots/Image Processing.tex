\documentclass[twocolumn]{article}
\usepackage{graphicx,float}
\author{My Penis}
\title{Chicken Chicken Chicken}



\newcommand{\dkepic}[2]{ %2 referes to the amount of "parameters" in this new "method"
	\begin{figure}[H] %the H signifies that the image will be put 'Here', and can only be used by using package 'float'
	\includegraphics[width=0.5\textwidth]{#1}
	\caption{#2}
	\label{#1}
	\end{figure}
}



\begin{document}
\maketitle

\section{other section}
\section{other section}
\section{other section}

\section{Image Processing}
This section's purpose is to explain the mechanisms, algorithms and mathematics behind the processing of images made by the Nao.
A brief introduction to the subsections follows before heading into the details.\\ \\
First the image is preprocessed. In this step the image is converted into a new image consisting only of green, yellow and white pixels. Any background noise is removed.\\
Second the fieldlines get isolated and using Hough Transforms the cornerpoints are extracted.\\
Third the goalposts get isolated and the location of the foot of each post - if visible - gets determined.\\
Lastly the distance gets calculated using the camera angle and height, the resolution of the image and the coordinates of the previously extracted landmarks.

\subsection{Image Preprocessing}
Before any kind of extraction can be made, the image first needs to be preprocessed. First a random sampling of pixels is made from which the average light intensity is calculated using the luminence from the HSL color-space.\\
Next the white, green and yellow (goal) pixels are extracted by converting them to the HSL color-space and by using the following thresholds and bounds:\\ \\
White:\\ $L_{p}$ $>$ 120 + ($L_{max}$-120) $\times$ $\frac{L_{avg}}{L_{max}}$ \\ \\
Green:\\ $G_{min}$ $\leq$ H $\leq$ $G_{max}$\\ \\
Yellow:\\ $Y_{min}$ $\leq$ H $\leq$ $Y_{max}$\\ \\
Where \\
$L_{max}$ = maximum luminence (usually 255),\\
$L_{p}$ = luminence of pixel, \\
$L_{avg}$ = previously approximated average luminence, \\
H = hue of pixel, and \\
$G_{min}$, $G_{max}$, $Y_{min}$, $Y_{max}$ = lower and upper bounds for the hue of green and yellow.\\ \\
The value of '120' has experimentally been shown to deliver the best results. The $G_{min}$ and $G_{max}$ are determined at the start of each run by taking a picture of the 'grass'. This image is then scanned for the minimum and maximum hue. $G_{min}$ will be set equal to this minimum hue minus some small number $\epsilon$ in order to deal with the occasional over- and underlighted areas on the field. The same is done for $G_{max}$.\\
$Y_{min}$ and $Y_{max}$ are determined similarly but without adding/substracting $\epsilon$, because (in the case of this project) there is only one goal and thus the difference in lighting is negligible.\\
At this point the original image(figure 4.1) now looks something like figure 4.2.\\
\dkepic{figure_IP1}{Raw image taken by Nao}
\dkepic{figure_IP2}{Image after color filter}
Before continuing preprocessing, the goalposts need to be extracted first.


\subsection{Goalpost Extraction}


\subsection{Fieldline Cornerpoint Extraction}
Before any fieldlines can be extracted the background noise needs to be removed. The details of this operation will not be explained here, but the general idea is to scan the color-filtered image from top to bottom for the color green. Once a safe amount of green pixels in a row has been encountered, 'erase' all the pixels above this point and continue to the next column [REF1]. To finish up we will remove any remaining green and yellow pixels because they aren't needed anymore. At this point the image will look something like figure 4.3.\\
\dkepic{figure_IP3}{Background noise removed}
There are quite some ways of obtaining the - in this example - three landmarks, but one in particular seems to be favored in the world of robotics [REF1, 2, etc...]: OpenCV's build-in Canny and Hough Transform methods. To illustrate Canny's use see figure 4.4.\\
\dkepic{figure_IP4}{Image after applying Canny}
After using the Canny method the Hough Transform is applied to the newly obtained image resulting in figure 4.5.\\
\dkepic{figure_IP5}{Image after applying Hough Transform}
The thicker lines are the lines generated by the Hough Transform method. The only thing left to do is finding the intersections of these lines and deleting intersections which are relatively close to each other in order to avoid double landmarks.\\
Now that all the landmarks have been extracted the distance to these can be calculated.


\subsection{Distance Calculation}




\end{document}


